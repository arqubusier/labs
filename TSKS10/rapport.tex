\documentclass[10pt,twocolumn]{article}

\usepackage{times}
\usepackage{mathptmx}
\usepackage{amsmath}
\usepackage{spverbatim}
\usepackage[T1]{fontenc}
\usepackage[utf8]{inputenc}
\usepackage[swedish]{babel}

\raggedbottom
\sloppy

\title{Laborationsrapport i TSKS10 \emph{Signaler, Information och Kommunikation}}

\author{Herman Lundkvist \\ herlu184, 930911-2770 }

\date{\today}

\begin{document}

\maketitle

\section{Inledning}

Denna rapport beskriver en lösningsgång för en laborationsuppgift som gavs i
kursen \emph{TSKS10} vårterminen 2015 på Tekniska högskolan vid Linköpings
universitet.

Uppgiften gick ut på extrahera en specifik I/Q-modulerad signal från en sänd
signal, $x(t)$, som innehöll flera I/Q-modulerade signaler. $x(t)$ ges av:
\begin{equation}
x(t) = x_i(t) cos(2 \pi f_ct) - x_q(t) sin(2 \pi f_c t) + z(t)
\label{e1}
\end{equation}
där $z(t)$ betecknar summan av de ointressanta signalerna.

Den eftersökta signalen hade några kända egenskaper. Dels hade signalen en
bärfrekvens $f_c$, som var en multipel av 19 Khz. Dels bestod $x_i$ respektive
$x_q$-signalerna av tre olika delar som vardera innehöll: en melodi, ett ordspråk
och vitt brus. Melodierna och ordpråket skiljde sig i
innehåll och längd mellan $x_i$ och $x_q$.

Signalen som mottogs från kanalen, $y$, såg pågrund av ekoeffekter ut på
följande vis:
\begin{equation}
    y(t)=x(t - \tau_1) + 0,9 x(t - \tau_2)
\end{equation}
där $\tau_1$ och $\tau_2$ är två tidskonstanter.

För att kunna utföra beräkningar med hjälp av dator, användes en samplad
variant av $y$. Den samplade signalen, $\hat{y}$, erhölls genom att först
filtrera y med ett idealt lågpassfilter och sedan sampla resultatet med
frekvensen 400 kHz.

De uppgifter som skulle lösas var:
\begin{enumerate}
\item att bestämma $f_c$;
\item att bestämma tidsfördröjningen $\tau_1-\tau_2$, under förutsättningen att
$\tau_2 > \tau_1$;
\item att identifiera ordpråket.
\end{enumerate}

\section{Metod}

Uppgiften l\"ostes p\aa\ f\"oljande s\"att...

\section{Resultat}

Den sökta informationen \"ar:
\begin{itemize}
\item B\"arfrekvensen f\"or nyttosignalen \"ar $f_c=...$
\item ...
\end{itemize}

\clearpage

\section*{Min Matlab-kod:}
\begin{spverbatim}
clear all
close all

for k=1:...
  ...
end

plot(...,...)
\end{spverbatim}

\end{document}
