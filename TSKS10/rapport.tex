\documentclass[10pt,twocolumn]{article}

\usepackage{times}
\usepackage{mathptmx}
\usepackage{amsmath}
\usepackage{spverbatim}
\usepackage[T1]{fontenc}
\usepackage[utf8]{inputenc}
\usepackage[swedish]{babel}

\raggedbottom
\sloppy

\title{Laborationsrapport i TSKS10 \emph{Signaler, Information och Kommunikation}}

\author{Herman Lundkvist \\ herlu184, 930911-2770 }

\date{\today}

\begin{document}

\maketitle

\section{Inledning}

Denna rapport beskriver en lösningsgång för en laborationsuppgift som gavs i
kursen \emph{TSKS10} vårterminen 2015 på Tekniska högskolan vid Linköpings
universitet.

Uppgiften gick ut på extrahera en specifik I/Q-modulerad signal från en en
kanal som innehöll flera I/Q-modulerade signaler.

%x här

Den eftersökta signalen hade några kända egenskaper. Dels hade signalen en
bärfrekvens fc, som var en multipel av 19 Khz. Dels bestod Xi respektive
Xq-signalerna av tre olika delar som vardera innehöll: en melodi, ett ordspråk
och vitt brus. Melodierna och ordpråket skiljde sig i
innehåll och längd mellan Xi och Xq.

Signalen som mottogs från kanalen, y, såg pågrund av ekoeffekter ut på
följande vis:

För att kunna utföra beräkningar med hjälp av dator, användes en samplad
variant av y, y-hatt. Y-hatt erhölls genom att först filtrera y med ett idealt
lågpassfilter och sedan sampla resultatet med frekvensen 400 kHz.

De uppgifter som skulle lösas var:
\begin{enumerate}
\item att bestämma fc;
\item att bestämma tidsfördröjningen t1-t2, under förutsättningen att
t2 > t1;
\item att identifiera ordpråket.
\end{enumerate}

\section{Metod}

Uppgiften l\"ostes p\aa\ f\"oljande s\"att...

\section{Resultat}

Den sökta informationen \"ar:
\begin{itemize}
\item B\"arfrekvensen f\"or nyttosignalen \"ar $f_c=...$
\item ...
\end{itemize}

\clearpage

\section*{Min Matlab-kod:}
\begin{spverbatim}
clear all
close all

for k=1:...
  ...
end

plot(...,...)
\end{spverbatim}

\end{document}
