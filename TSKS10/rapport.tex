\documentclass[10pt,twocolumn,a4paper]{article}

\addtolength{\textheight}{3cm}
\addtolength{\voffset}{-1.5cm}
\usepackage{times}
\usepackage{mathptmx}
\usepackage{amsmath}
\usepackage{mathtools}
\usepackage{spverbatim}
\usepackage{graphicx}
\usepackage[T1]{fontenc}
\usepackage{lmodern}
\usepackage[utf8]{inputenc}
\usepackage[swedish]{babel}
\usepackage{xspace}
\DeclareUnicodeCharacter{00A0}{ }
\sloppy
\raggedbottom

\title{Laborationsrapport i TSKS10 \emph{Signaler, Information och Kommunikation}}

\author{Herman Lundkvist \\ herlu184, 930911-2770 }

\date{\today}

\begin{document}

\newcommand{\yhat}{$\hat{y}[n]$\xspace}

\maketitle

\section{Inledning}

Denna rapport beskriver en lösningsgång för en laborationsuppgift som gavs i
kursen \emph{TSKS10} vårterminen 2015 på Tekniska högskolan vid Linköpings
universitet.

Uppgiften gick ut på att extrahera en specifik I/Q-modulerad signal, $x_t(t)$,
från en sänd signal, $x(t)$, som gas av:
\begin{equation}
x(t) = x_i(t) \cos(2 \pi f_ct) - x_q(t) \sin(2 \pi f_c t) + z(t)
\label{e1}
\end{equation}
där $x_i(t)$ och $x_q(t)$ utgör $x_t(t)$ och $z(t)$ betecknar
summan av två ointressanta signaler.

Signalen $x_t(t)$ hade följande kända egenskaper: dess bärfrekvens, $f_c$,
var en multipel av 19 kHz; och dess komponenter, $x_i$ respektive $x_q$,
var båda uppdelade i tre delar. De tre delarna innehöll: en melodi, ett
ordspråk och vitt brus, där melodierna och ordpråken skiljde sig åt i
innehåll och längd mellan $x_i$ och $x_q$.

Den signal som faktiskt mottogs och kunde studeras, $y$, såg på grund av
ekoeffekter ut på följande vis:
\begin{equation}
    y(t)=x(t - \tau_1) + 0,9 x(t - \tau_2)
\end{equation}
där $\tau_1$ och $\tau_2$ är två tidskonstanter.

För att kunna utföra beräkningar med hjälp av dator, filterades $y$ 
med ett idealt lågpassfilter och samplades med frekvensen 400 kHz.
Den samplade signalen, \yhat sparades sedan i wav-format.

De uppgifter som skulle lösas var:
\begin{enumerate}
\item att bestämma $f_c$;
\item att bestämma tidsfördröjningen $\tau_1-\tau_2$, under förutsättningen att
$\tau_2 > \tau_1$;
\item att identifiera ordpråken.
\end{enumerate}

\section{Metod}

För att lösa uppgiterna, användes ett Matlab-skript som laddade in wav-filen,
och utförde ett antal beräkningar.
\subsection{Filtrering av eko}

Det första som gjordes var att bestämma $\tau_1$ - $\tau_2$, för att därmed
kunna filtrera bort ekot från $\hat{y}[n]$. Detta gjordes genom att studera
mångtydighetsfunktionen av $y$:
\begin{multline}
    r(\tau) = \int_{-\infty}^{\infty}\!y(t)y(t+\tau)\, \mathrm{d}t \\ 
    = 1,86\left\{x(t) \ast x(-t)\right\}(\tau)\\ + 
    0,9\left\{x(-t) \ast x(t+\tau_1-\tau_2))\right\}(\tau)\\ + 
    0,9\left\{x(-t) \ast x(t+\tau_2-\tau_1))\right\}(\tau)
    \label{e3}
\end{multline}
Här utnyttjades att integralen hade $\pm\infty$ som integrationsgränser,
vilket gjorde att integralens värde ej påverkades om integranden
tidsförskjöts med $\tau_1$.

Från (\ref{e3}) kan man dra slutsatsen att $r(\tau)$ innehåller tre
kraftiga pikar: en stor pik vid $\tau=0$, mitt emellan två mindre
pikar på avstånden $\pm(\tau_2 - \tau_1)$ från den större. Dessa pikar uppstår
då de båda komponeterna i respekive faltning överlappar som mest. Ett liknade
resonemang kan göras för den diskreta \yhat.

\begin{figure}
    \includegraphics[trim = 0 80mm 0 90mm, clip, width=\linewidth]{fig1.pdf}
    \caption{
        Korrelationen $r[n]$ 
        \label{fig:z}
    }
\end{figure}

mångtydighetsfunktionen av \yhat  plottades, se figur \ref{fig:z}, och
utifrån denna beräknades $\tau_2 - \tau_1$ genom att först beräkna
$n_\Delta$, antalet sampel mellan den mittersta och den högra av ovannämnda
pikar. Detta räknades sedan om till en tidsdifferens genom att multiplicera
resultatet sampeltiden.

Tidsdifferensen användes sedan för att filtrera ut ekot. Detta gjordes i
tidsdomänen genom att gå igenom varje sampelvärde från $n_\Delta$ till det
sista, och för varje sampelvärde, $n_i$ sätta 
\begin{equation}
x[n_i] \coloneqq x[n_i] - 0,9x[n_i - n_\Delta] 
\label{e4}
\end{equation}
där $x[n] \coloneqq \hat{y}[n]$ från början.

Operationen (\ref{e4}) fungerar, eftersom att ekot börjar (bortsett från
noll-värden i början av signalen som ej påverkar resultatet) $n_\Delta$
sampel efter själva signalen. Det finns alltså ett fönster av $n_\Delta$
sampel i början av signalen som saknar eko. Genom att värden från detta fönster
subtraheras från värden $n_\Delta$ sampel senare, kancelleras ekot. Eftersom
att man tilldelar $x$ det ekofria värdet samtidigt som det används
i uträkningen, flyttar man detta fönster framför sig tills man nått slutet av
signalen.

\subsection{Bestämning av $f_c$}
%crop arguments: left bottom top right
\begin{figure}[width=\textwidth]

    \includegraphics[trim = 0 80mm 0 90mm, clip, width=\linewidth]{fig2.pdf}
    \caption{
        Den diskreta fouriertransformen av $x[n]$ 
        \label{fig:X}
    }
\end{figure}

\begin{figure}[width=\textwidth]
    \includegraphics[trim = 0 80mm 0 90mm, clip, width=\linewidth]{fig3.pdf}
    \caption{
        Samtliga frekvensband från $X[F]$ bandpassfiltrerade och inverstransformerade. 
        \label{fig:xt}
    }
\end{figure}

Genom att plotta $X[F]$, den diskreta fouriertransformen av $x[n]$, kunde tre
tydliga frekvensband med centra kring $F$, som motsvarade 38 kHz, 95 kHz, och
152 kHz ($f = F f_s$) observeras, se figur \ref{fig:X}. Banden undersöktes var
för sig i tidsdomänen. Först filtrerades banden med ett idealt lågpassfilter,
med en bandbredd sådan att majoriteten av signalenergin för varje band rymdes.
Sedan inverstransformerades resultaten vilket gav signalerna $x_1[n]$, $x_2[n]$
och $x_3[n]$, se figur \ref{fig:xt}. Från samma figur blev det tydligt att
$x_3$ var den eftersökta signalen, eftersom att den var den enda med tre
distinkta områden där det sista området påminde om vitt brus. Frekvensen, 152
kHz, som var i centrum för frekvensbandet hos detta område noterades, och
antogs vara $f_c$. 

\subsection{Identifiering av ordspråken}

För att identifiera ordspråket genomfördes en I/Q-demodulering av $x[n]$,
enligt:
\begin{multline}
    x_i[n] = \mathcal{H}_{B/2}^{LP} \{ 2 x[n] \cos[2 \pi f_c n] \} \\
    x_q[n] = \mathcal{H}_{B/2}^{LP} \{ -2 x[n] \sin[2 \pi f_c n] \}
    \label{e5}
\end{multline}
där $B$ valdes till ett värde som var tillräckligt stort för att rymma en
majoritet av signalenergin.

Det uppstod dock ett problem i och med att $x[n]$ innehöll en tidsförskjutning
$n_{\tau_1}$, vilket medförde att de komponenter man fick ut från (\ref{e5})
innehöll linjärkombinationer av $x_i[n]$ och $x_q[n]$. Detta kan inses genom
att sätta $n := n - n_{\tau_1}$ i (\ref{e5}) och använda subtraktionsatsen för
sinus och cosinus.

Eftersom att det inte fanns något enkelt sätt att bestämma $n_{\tau_1}$, togs
denna fram genom att experimentera med olika värden mellan $0$ och $2 \pi$.
Efter att några värden provats kunde man dra slutsatsen att en av melodierna
var kortare en den andra och följdes av några sekunders tystnad. Värdet på
$n_{\tau_1}$ finjusterades därför tills överhörningen från den andra signalen
var minimal under denna period av tystnad.

Sist av allt nedsamplades $x_i$ respektive $x_q$ med en faktor 10 och spelades
upp varpå ordspråken kunde höras.
\section{Resultat}

Laborationen gav följande resultat:
\begin{enumerate}
\item Bärfrekvensen $f_c$ var 152 KHz. 
\item Tidsdifferensen $\tau_2 - \tau_1$ var 0.39 s.
\item Ordspråken löd dels "inget ont som inte har något gott med sig", dels "väck inte den björn som sover".
\end{enumerate}

\clearpage

\section*{Min Matlab-kod:}
\begin{spverbatim}
%%Initialize
N = 7.8*10^6;
[y,Fs] = wavread('signal-herlu184.wav');
f = Fs*linspace(0,1/2,N/2);
t = 0:1/Fs:(N-1)*1/Fs;

%% 
% Determine the tau2-tau1 by studying the 
% correlation of y with itself. The Peak
% values are derived from the plot in 
% this section
% 
y_c = xcorr(y, y);
plot(y_c);

lpeak = 7.644*10^6;
mpeak = 7.800*10^6;
rpeak = 7.956*10^6;

%tau2 - tau1 in # of samples
n_delta = rpeak-mpeak;
tau_diff = n_delta*1/Fs;

%%
% Filter out echo in the time domain.
% x is the signal with the echo removed
%
x = y;
for i = n_delta+1:length(x)
    x(i) = x(i) - 0.9*x(i-n_delta);
end

x_c1 = xcorr(x, x);
%The correlation shows that the left and
%right peaks have been removed
plot(x_c1)

%%
% Determine fc
X = fft(x);
%The plot shows three distinct bands at
%frequencies with multiples of 19 kHz
plot(f, abs(X(1:end/2)))

%Filtering out each band and inverse
%transforming it shows that the
%heighest band matches the signal
%description (three distinctive parts,
%the last one being white noise)
ze = zeros(1, N);
X_target1 = ze;
X_target2 = ze;
X_target3 = ze;
X_target1(0.2/2*N/2:0.6/2*N/2) = X(0.2/2*N/2:0.6/2*N/2);
X_target2(0.7/2*N/2:1.1/2*N/2) = X(0.7/2*N/2:1.1/2*N/2);
X_target3(1.3/2*N/2:1.7/2*N/2) = X(1.3/2*N/2:1.7/2*N/2);
X_target = X_target3;
x_target = ifft(X_target, 'symmetric');

plot(x_target)

%fc i determined by looking at the 
%centrum of the heighest band of X
fc = 152*10^3;

%% 
% Demodulate
%
% The delay of x(t) results in a phase
% shift in the xi, and xq components.
% This value lies somewhere between
% zero and pi/2. The phase shift below
% was derived by testing different 
% values until the message could be heard.
%
phase_shift = 0.8;
xi_mixer = cos(2*pi*fc*t+phase_shift);
xq_mixer = sin(2*pi*fc*t+phase_shift);
xi_ = 2*xi_mixer.*x_target;
xq_ = -2*xq_mixer.*x_target;
Xi_ = fft( xi_ );
Xq_ = fft( xq_ );

%lowpass filter
Xi = zeros(1,N);
Xq = zeros(1,N);
Xi(1:0.4/2*N/2) = Xi_(1:0.4/2*N/2);
Xq(1:0.4/2*N/2) = Xq_(1:0.4/2*N/2);
xi = ifft(Xi, 'symmetric');
xq = ifft(Xq, 'symmetric');

plot(f, abs(Xq(1:end/2)));

Fs_ = Fs/10

soundsc(xi(1:20:end), Fs/20)
pause
soundsc(xq(1:20:end), Fs/20)  

%Xi "Inget ont som inte har något gott med
%sig."
%Xq "Väck inte den björn som sover."
\end{spverbatim}

\end{document}
